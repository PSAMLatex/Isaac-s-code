\documentclass[11pt, oneside]{article}   	% use "amsart" instead of "article" for AMSLaTeX format
\usepackage{geometry}                		% See geometry.pdf to learn the layout options. There are lots.
\geometry{letterpaper}                   		% ... or a4paper or a5paper or ... 
%\geometry{landscape}                		% Activate for rotated page geometry
\usepackage[parfill]{parskip}    		% Activate to begin paragraphs with an empty line rather than an indent
\usepackage{graphicx}				% Use pdf, png, jpg, or eps§ with pdflatex; use eps in DVI mode
								% TeX will automatically convert eps --> pdf in pdflatex		
\usepackage{mathtools,amssymb,tcolorbox}

%SetFonts

%SetFonts


%\title{}
%\author{Isaac Baer}
%\date{}							% Activate to display a given date or no date

\begin{document}
%\section*{}
%\subsection{}
\begin{tcolorbox}[title=\textbf{1.1.12}]
Solve the system
\[x_1- 3x_2+4x_3=-4\]
\[3x_1-7x_2+7x_3=-8\]
\[-4x_1+6x_2-x_3=7\]
\end{tcolorbox}
The associated augmented matrix for this system is 
\[
\begin{pmatrix*}[r]
1&-3&4&-4\\
3&-7&7&-8\\
-4&6&-1&7\end{pmatrix*}
\]

By a series of row operations we find 

\[
\begin{pmatrix*}[r]
1&-3&4&-4\\
3&-7&7&-8\\
-4&6&-1&7\end{pmatrix*}
\sim \begin{pmatrix*}[r]
1&-3&4&4\\
0&2&-5&4\\
0&-6&15&-9\end{pmatrix*}
\sim \begin{pmatrix*}[r]
1&-3&4&-4\\
0&2&-5&4\\
0&0&0&3\end{pmatrix*}
\]

Since there is a pivot position in the last column of the echelon matrix, the equation $0=3$ is implied, which is absurd. Therefore the system of linear equations is inconsistent and there is no solution.

\clearpage
\begin{tcolorbox}[title=\textbf{1.4.14}]
\begin{center}
Let $\textbf{u}=\begin{pmatrix*}[r]
2\\
-3\\
2\end{pmatrix*}$ and $A=\begin{pmatrix*}[r]
5&8&7\\
0&\hphantom{-}1&-1\\
1&3&0\end{pmatrix*}$. 
\end{center}

\vspace{5 mm}

Is \textbf{u} in the subset of $\mathbb{R}^3$ spanned by the columns of A? Why or why not?
\end{tcolorbox}

By definition, \textbf{u} is in the subset of $\mathbb{R}^3$ spanned by the columns of A if \textbf{u} is a linear combination of the columns of A, or more explicitly if there exist scalars $x_1, x_2,$ and $x_3$ such that

\[
x_1
\begin{pmatrix*}[r]
5\\
0\\
1\end{pmatrix*}
+ x_2
\begin{pmatrix*}[r]
8\\
1\\
3\end{pmatrix*}
+x_3
\begin{pmatrix*}[r]
7\\
-1\\
0\end{pmatrix*}
=
\begin{pmatrix*}[r]
2\\
-3\\
2\end{pmatrix*}
\]

This vector equation has a solution if and only if there exists a solution to the linear system associated with the augmented matrix

\[
\begin{pmatrix*}[r]
5&8&7&2\\
0&1&-1&3\\
1&3&0&2\end{pmatrix*}
\]

To check whether a solution exists, we perform the following series of row operations to arrive at a matrix in echelon form.

\[
\begin{pmatrix*}[r]
5&8&7&2\\
0&1&-1&3\\
1&3&0&2\end{pmatrix*}
\sim
\begin{pmatrix*}[r]
1&3&0&2\\
0&1&-1&3\\
5&8&7&2\\
\end{pmatrix*}
\sim
\begin{pmatrix*}[r]
1&3&0&2\\
0&1&-1&3\\
0&-7&7&-8
\end{pmatrix*}
\sim
\begin{pmatrix*}[r]
1&3&0&2\\
0&1&-1&3\\
0&0&0&13
\end{pmatrix*}
\]

Since the last column of this echelon matrix has a pivot position, the associated linear system must be inconsistent. Therefore, the previously stated vector equation must have no solution. Thus \textbf{u} is not an element of the subset of $\mathbb{R}^3$ spanned by the columns of A.









\end{document}  