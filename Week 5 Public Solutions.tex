\documentclass[11pt, oneside]{article}   	% use "amsart" instead of "article" for AMSLaTeX format
\usepackage{geometry}                		% See geometry.pdf to learn the layout options. There are lots.
\geometry{letterpaper}                   		% ... or a4paper or a5paper or ... 
%\geometry{landscape}                		% Activate for rotated page geometry
\usepackage[parfill]{parskip}    		% Activate to begin paragraphs with an empty line rather than an indent
\usepackage{graphicx}				% Use pdf, png, jpg, or eps§ with pdflatex; use eps in DVI mode
								% TeX will automatically convert eps --> pdf in pdflatex		
\usepackage{amssymb}
\usepackage{mathtools}
\usepackage{tcolorbox}	
\usepackage{braket}			

\begin{document}

\begin{center}
\begin{tcolorbox}[title=1.14 \hfill Isaac Baer , halign title=center]

It is known that there is a 36\% probability of obtaining $S_z=\hbar/2$ and therefore a 64\% probability of obtaining $S_z=-\hbar/2$ if a measurement of $S_z$ is carried out on a certain spin-$\frac{1}{2}$ particle. In addition, it is known that the probability of finding the particle with $S_x=\hbar/2$, that is in the state $\ket{+\textbf{x}}$, is 50\%. Determine the state of the particle as completely as possible from this information.
\end{tcolorbox}
\end{center}

Let's call the state of this particle $\ket{\psi}$. We can express $\ket{\psi}$ as a superposition of the states $\ket{+\textbf{z}}$ and $\ket{-\textbf{z}}$ with respective amplitudes $c_+$ and $c_-$ like so.

\[\ket{\psi}=c_+\ket{+\textbf{z}} + c_-\ket{\textbf{z}}\]

The amplitudes $c_+$ and $c_-$ must be able to square to become the probabilities given for the $S_z$ measurement, so $c_+^2=.36$ and $c_-^2=.64$. We interpret this squaring to mean each amplitude is multiplied by its complex conjugate. This gives us four possible values for each amplitude, two real and two imaginary.

\[
c_+=\pm0.6 \textbf{ or } c_+=\pm0.6i
\]\[
c_-=\pm0.8 \textbf{ or } c_-=\pm0.8i
\]

Since $\ket{\psi}$ could be any combination of these, we have 16 possible states the particle could be in. We can narrow this number down using the fact that the probability of finding the particle with $S_x=\hbar/2$ is 50\%. This tells us that when the measurement $\bra{+\textbf{x}}$ is made on the state $\ket{\psi}$, the resultant probability should be 1/2. so

\[
\braket{+\textbf{x}|\psi}^2=\frac{1}{2}
\]

$\ket{+\textbf{x}}$ expressed in the most general form as a superposition of z states is 

\[
\ket{+\textbf{x}}=\frac{e^{i\delta_+}}{\sqrt{2}}\ket{+\textbf{z}}+\frac{e^{i\delta_-}}{\sqrt{2}}\ket{-\textbf{z}}
\]

Thus 

\[
\bra{+\textbf{x}}=\frac{e^{i\delta_+}}{\sqrt{2}}\bra{+\textbf{z}}+\frac{e^{i\delta_-}}{\sqrt{2}}\bra{-\textbf{z}}
\]

Note that since the phase $\delta$ is arbitrary, where taking the complex conjugate of the amplitudes would normally have entailed a negative multiple of the phase, here I have allowed the negative to remain implicit. 
 
 If we operate $\bra{+\textbf{x}}$ on $\ket{\psi}$ we get
 
 \[
 \braket{+\textbf{x}|\psi}= \frac{e^{i\delta_+}}{\sqrt{2}}c_++\frac{e^{i\delta_-}}{\sqrt{2}}c_-
 \]
 
 I can factor out the $\frac{1}{\sqrt{2}}$ to get 
 
 \[ 
 \braket{+\textbf{x}|\psi}= \frac{1}{\sqrt{2}}\big(e^{i\delta_+}c_++e^{i\delta_-}c_-\big)
 \]
 
 Since I intend to square this amplitude I can do away with the phase information, since the magnitude of $e^{i\delta}$ is one no matter what. 
 
 \[
  \braket{+\textbf{x}|\psi}= \frac{1}{\sqrt{2}}\big(c_++c_-\big)
 \]
 
 Since $\braket{+\textbf{x}|\psi}^2=\frac{1}{2}$, I know 
 
 \begin{equation*}
 \frac{1}{2}=\Big(\frac{1}{\sqrt{2}}\big(c_++c_-\big)\Big)^2
 =\frac{1}{2}\big(c_++c_-\big)^2
 \end{equation*}
 
And therefore

\[
\big(c_++c_-\big)^2=1
\]
 
 I already know from before that 
 
 \[
 c_+^2+c_-^2=1
 \]
 
 So it must be the case that 
 
 \[
 \big(c_++c_-\big)^2= c_+^2+c_-^2
 \]
 
 This is only possible if $c_++c_-$ is complex. Or in other words if $c_+$ is real, then $c_-$ must be imaginary or vice versa. 
 
 \clearpage
 
 \begin{center}
 \begin{tcolorbox}[title=5.1.14 \hfill Isaac Baer, halign title=center]
 \[
 \text{Let A}= \begin{pmatrix*}[r]
 1&0&-1\\
 1&-3&0\\
 4&-13&1
 \end{pmatrix*}
 \]
 Find a basis for the eigenspace corresponding to the eigenvalue $\lambda=-2$
 \end{tcolorbox}
 \end{center}
 
 I will construct the basis for the eigenspace by finding the basis for the nullspace of the matrix (A-$\lambda$I)
 
 \[
 \big(A-\lambda I\big)= \begin{pmatrix*}[r]
 1&0&-1\\
 1&-3&0\\
 4&-13&1
 \end{pmatrix*}
- \begin{pmatrix*}[r]
-2&0&0\\
0&-2&0\\
0&0&-2
\end{pmatrix*}
=\begin{pmatrix*}[r]
 3&0&-1\\
 1&-1&0\\
 4&-13&3
 \end{pmatrix*}
\]

To find the null space I augment this matrix with the 0 vector in $\mathbb{R}^3$ and find the associated reduced echelon matrix.

\[
\begin{pmatrix*}[r]
3&0&-1&0\\
1&-1&0&0\\
4&-13&3&0
\end{pmatrix*}
\sim 
\begin{pmatrix*}[r]
1&-1&0&0\\
4&-13&3&0\\
3&0&-1&0
\end{pmatrix*}
\sim \begin{pmatrix*}[r]
1&-1&0&0\\
 0&-9&3&0\\
 0&3&-1&0
 \end{pmatrix*}
 \sim \begin{pmatrix*}[r]
1&-1&0&0\\
 0&-9&3&0\\
 0&0&0&0
 \end{pmatrix*}
 \]
 \[
 \sim \begin{pmatrix*}[r]
1&-1&0&0\\
 0&1&\frac{-1}{3}&0\\
 0&0&0&0
 \end{pmatrix*}
 \sim \begin{pmatrix*}[r]
1&0&\frac{-1}{3}&0\\
 0&1&0&0\\
 0&0&0&0
 \end{pmatrix*}
 \]
 
 We can now see that solutions to the linear system associated with this matrix are of the form 
 
 \[
 \begin{pmatrix*}
 x_1\\
 x_2\\
 x_3
 \end{pmatrix*}
 =
 \begin{pmatrix*}
 \frac{-1}{3}x_3\\
 0\\
 x_3
 \end{pmatrix*}
 =
 x_3 \begin{pmatrix*}
 \frac{-1}{3}\\
 0\\
 1
 \end{pmatrix*}
 \]

Therefore all solutions to the matrix equation Ax=$\lambda$x are scalar multiples of the vector

\[
\begin{pmatrix*}
 \frac{-1}{3}\\
 0\\
 1
 \end{pmatrix*}
 \]
 
 and this vector serves as a basis for the eigenspace associated with $\lambda$.

 
\end{document}  