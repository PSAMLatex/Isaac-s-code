\documentclass[11pt]{amsart}
\usepackage{geometry}                % See geometry.pdf to learn the layout options. There are lots.
\geometry{letterpaper}                   % ... or a4paper or a5paper or ... 
\usepackage[parfill]{parskip}    % begin paragraphs with an empty line rather than an indent
\usepackage{graphicx}
\usepackage{amssymb}
\usepackage{mathtools}          % like amsmath but gives better matrix options + other cool stuff
\usepackage{tcolorbox}           % Lets you make boxes
\usepackage{epstopdf}
\usepackage[utf8]{inputenc}        %provides proof and theorem environments (just proofs given w/ mathtools?)
\usepackage[english]{babel}       %same
\DeclareGraphicsRule{.tif}{png}{.png}{`convert #1 `dirname #1`/`basename #1 .tif`.png}

\begin{document}
\begin{tcolorbox}[title=2.1.27 \hfill Isaac Baer , halign title=center] % 1st probem
\begin{center}
Let $\textbf{u}=
\begin{pmatrix*}[r]
-2\\
3\\
-4
\end{pmatrix*}$
and $\textbf{v}=
\begin{pmatrix*}[r]
a\\
b\\
c
\end{pmatrix*}
$.
\end{center} 

\vspace{10mm}

Compute $\textbf{u}^T\textbf{v}$, $\textbf{v}^T\textbf{u}$, $\textbf{uv}^T$, and $\textbf{vu}^T$
\end{tcolorbox}

Since \textbf{u} and \textbf{v} are both $3\times1$ matrices, Their transposes will be $1\times3$ matrices with the same first entry. We can see that if we multiply one by the transpose of the other, the product will always be defined.

$\textbf{u}^T\textbf{v}$ looks like 
\[
\begin{pmatrix*}[r]
-2&3&-4
\end{pmatrix*}
\begin{pmatrix*}[r]
a\\
b\\
c
\end{pmatrix*}
\]

This is just like any matrix multiplied by a vector. We multiply the rows of the vector by the columns of the matrix, and then sum the products to get

\[
\begin{pmatrix}
-2a+3b-4c\
\end{pmatrix}
\]

Note (for later) that no matter the size of the original vectors, this product will always be a $1\times1$ matrix.

$\textbf{v}^T\textbf{u}$ looks similar

\[
\begin{pmatrix*}[r]
a&b&c
\end{pmatrix*}
\begin{pmatrix*}[r]
-2\\
3\\
-4
\end{pmatrix*}
\]

And can be computed in the same way to find

\[
\begin{pmatrix}
-2a+3b-4c\
\end{pmatrix}
\]

$\textbf{uv}^T$ looks a bit different 

\[
\begin{pmatrix*}[r]
-2\\
3\\
-4
\end{pmatrix*}
\begin{pmatrix*}[r]
a&b&c
\end{pmatrix*}
\]

To compute this we use the definition of matrix multiplication given on page 97. This gives us a $3\times3$ matrix with columns $
\begin{pmatrix*}[r]
-2\\
3\\
-4
\end{pmatrix*}a$, $
\begin{pmatrix*}[r]
-2\\
3\\
-4
\end{pmatrix*}b$, $
\begin{pmatrix*}[r]
-2\\
3\\
-4
\end{pmatrix*}c$. which simplifies to 

\[
\begin{pmatrix*}[r]
-2a&-2b&-2c\\
3a&3b&3c\\
-4a&-4b&-4c
\end{pmatrix*}
\]

$\textbf{vu}^T$ looks like 

\[
\begin{pmatrix*}[r]
a\\
b\\
c
\end{pmatrix*}
\begin{pmatrix*}[r]
-2&3&-4
\end{pmatrix*}
\]

and can be computed in the same manner to get

\[
\begin{pmatrix*}[r]
-2a&3a&-4a\\
-2b&3b&-4b\\
-2c&3c&-4c
\end{pmatrix*}
\]

\clearpage                                                          % 2nd problem

\begin{tcolorbox}[title=2.1.28 \hfill Isaac Baer]
If $\textbf{u}$ and $\textbf{v}$ are in $\mathbb{R}^n$, how are $\textbf{u}^T\textbf{v}$ and $\textbf{v}^T\textbf{u}$ related? How are $\textbf{uv}^T$ and $\textbf{vu}^T$ related?

\end{tcolorbox}

\vspace{1cm}

\begin{center}                                                                 % 1st answer
\begin{tcolorbox}[width=\linewidth/4, sharp corners]
\begin{center}
$\textbf{u}^T\textbf{v}=\textbf{v}^T\textbf{u}$
\end{center}
\end{tcolorbox}
\end{center}

\renewcommand\qedsymbol{$\blacksquare$} % changes QED symbol to a black square

\begin{proof}
Remember $\textbf{u}^T\textbf{v}$ is a $1\times1$ matrix. Therefore it must equal its transpose.
\[
\textbf{u}^T\textbf{v}=\left(\textbf{u}^T\textbf{v}\right)^T
\]
Since $\textbf{u}$ and $\textbf{v}$ are sized such that the products listed are defined, we can simplify the right side using Theorem 3, 
\begin{equation*}
\begin{split}
\left(\textbf{u}^T\textbf{v}\right)^T & = \left(\textbf{v}\right)^T\left(\textbf{u}^T\right)^T\\
& = \textbf{v}^T\textbf{u}
\end{split}
\end{equation*}

Therefore $\textbf{u}^T\textbf{v}=\textbf{v}^T\textbf{u}$.
\end{proof}

\vspace{1cm}

\begin{center}                                                                  % 2nd answer
\begin{tcolorbox}[width=\linewidth/4, sharp corners]
\begin{center}
$\textbf{vu}^T=\left(\textbf{uv}^T\right)^T$
\end{center}
\end{tcolorbox}
\end{center}

\begin{proof}
This follows similar reasoning to the previous proof
By Theorem 3, 

\begin{equation*}
\begin{split}
\left(\textbf{uv}^T\right)^T & =\left(\textbf{v}^T\right)^T\left(\textbf{u}\right)^T\\
& = \textbf{vu}^T
\end{split}
\end{equation*}

Thus $\textbf{vu}^T=\left(\textbf{uv}^T\right)^T$.

\end{proof}

\end{document}  