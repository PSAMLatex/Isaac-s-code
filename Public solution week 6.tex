\documentclass[11pt]{amsart}
\usepackage{geometry}                % See geometry.pdf to learn the layout options. There are lots.
\geometry{letterpaper}                   % ... or a4paper or a5paper or ... 
\usepackage[parfill]{parskip}    % Activate to begin paragraphs with an empty line rather than an indent
\usepackage{graphicx}
\usepackage{amssymb}
\usepackage{epstopdf}
\usepackage{mathtools}
\renewcommand\qedsymbol{$\blacksquare$}     %makes the QED symbol a black box instead of a white one
\usepackage{tcolorbox}     % makes boxes
\usepackage{braket}       % lets you do easy dirac notation

\DeclareGraphicsRule{.tif}{png}{.png}{`convert #1 `dirname #1`/`basename #1 .tif`.png}

\begin{document}
\begin{center}                                                                                     %1st problem
\begin{tcolorbox} [title=2.7 \hfill Isaac Baer, halign title=center]

Work out the matrix representations of the projection opreators $\hat{P}_+=\ket{+\textbf{z}}\bra{+\textbf{z}}$ and $\hat{P}_-=\ket{-\textbf{z}}\bra{-\textbf{z}}$ using the states $\ket{+\textbf{y}}$ and $\ket{-\textbf{y}}$ of a spin-$\frac{1}{2}$ particle as a basis. Check that the results (2.51) and (2.52) are satisfied using these matrix representations.

\end{tcolorbox}
\end{center}

\vspace{10mm}

We can represent $\hat{P}_+$ in the $\ket{+\textbf{z}}-\ket{-\textbf{z}}$ basis with the matrix

\[
\hat{P}_+=
\begin{pmatrix*}[r]
1&0\\
0&0
\end{pmatrix*}
\]

The problem is essentially asking us to find a way to use this projection on a state given in the $\ket{+\textbf{y}}-\ket{-\textbf{y}}$ basis. An $\textit{Essence of Linear Algebra}$ video we watched in class gave me a lot of help with this one. In order to perform a linear transformation of a particle in the $\ket{+\textbf{y}}-\ket{-\textbf{y}}$ basis when the tranformation is given in the $\ket{+\textbf{z}}-\ket{-\textbf{z}}$ basis, we need to find the change of coordinates matrix that takes us from y to z, then do the transformation, then use the change of basis matrix that takes us from z back to y. Townsend solidifies this in equation (2.98). In this case we can write

\[
\hat{P}_+\xrightarrow[\ket{+\textbf{y}}-\ket{-\textbf{y}}\text{basis}]{}\mathbb{S}^{\dagger}\hat{P}_+\mathbb{S} %notice blank argument for \xrightarrow
\]

Where $\mathbb{S}$ is the change-of-basis matrix that turns y coordinates into z coordinates. 

We know 

\[
\ket{+\textbf{y}}=\frac{1}{\sqrt{2}}\ket{+\textbf{z}}+\frac{i}{\sqrt{2}}\ket{-\textbf{z}}
\]
and
\[
\ket{-\textbf{y}}=\frac{1}{\sqrt{2}}\ket{+\textbf{z}}-\frac{i}{\sqrt{2}}\ket{-\textbf{z}}
\]

$\mathbb{S}$ has columns composed of the z coordinates of $\ket{+\textbf{y}}$ and $\ket{-\textbf{y}}$, so 

\[
\mathbb{S}=\frac{1}{\sqrt{2}}
\begin{pmatrix*}[r]
1&1\\
i&-i
\end{pmatrix*}
\]

$\mathbb{S}^{\dagger}$ follows quickly by taking the conjugate transpose of $\mathbb{S}$ to find

\[
\mathbb{S}^{\dagger}=\frac{1}{\sqrt{2}}
\begin{pmatrix*}[r]
1&1\\
i&-i
\end{pmatrix*}
\]

Thus 

\[
\hat{P}_+\xrightarrow[\ket{+\textbf{y}}-\ket{-\textbf{y}}\text{basis}]{}
\frac{1}{\sqrt{2}}
\begin{pmatrix*}[r]
1&1\\
i&-i
\end{pmatrix*}
\begin{pmatrix*}[r]
1&0\\
0&0
\end{pmatrix*}
\frac{1}{\sqrt{2}}
\begin{pmatrix*}[r]
1&1\\
i&-i
\end{pmatrix*}
\]

\vspace{6mm}

We evaluate this product to find

\[
\hat{P}_+\xrightarrow[\ket{+\textbf{y}}-\ket{-\textbf{y}}\text{basis}]{}
\frac{1}{2}
\begin{pmatrix*}[r]
1&1\\
1&1
\end{pmatrix*}
\]

\vspace{5mm}

We can follow a similar process to find $\hat{P}_-=\ket{-\textbf{z}}\bra{-\textbf{z}}$ using the  $\ket{+\textbf{y}}-\ket{-\textbf{y}}$ basis, by evaluation the product

\[
\hat{P}_-\xrightarrow[\ket{+\textbf{y}}-\ket{-\textbf{y}}\text{basis}]{}
\frac{1}{\sqrt{2}}
\begin{pmatrix*}[r]
1&1\\
i&-i
\end{pmatrix*}
\begin{pmatrix*}[r]
0&0\\
0&1
\end{pmatrix*}
\frac{1}{\sqrt{2}}
\begin{pmatrix*}[r]
1&1\\
i&-i
\end{pmatrix*}
\]

to find 

\[
\hat{P}_-\xrightarrow[\ket{+\textbf{y}}-\ket{-\textbf{y}}\text{basis}]{}
\frac{1}{2}
\begin{pmatrix*}
1&-1\\
-1&1
\end{pmatrix*}
\]

All that remains is to be shown are the following results from (2.51) and (2.52):

\[
\hat{P}_+^2=\hat{P}_+
\]

\begin{proof}

\begin{equation*}
\begin{split}
\hat{P}_+^2 & =
\frac{1}{2}
\begin{pmatrix*}[r]
1&1\\
1&1
\end{pmatrix*}
\times
\frac{1}{2}
\begin{pmatrix*}[r]
1&1\\
1&1
\end{pmatrix*}\\
&=
\frac{1}{4}
\begin{pmatrix*}[r]
2&2\\
2&2
\end{pmatrix*}\\
&=
\frac{1}{2}
\begin{pmatrix*}[r]
1&1\\
1&1
\end{pmatrix*}\\
&=
\hat{P}_+
\end{split}
\end{equation*}

\end{proof}

\[
\hat{P}_-^2=\hat{P}_-
\]

\begin{proof}

\begin{equation*}
\begin{split}
\hat{P}_-^2 & =
\frac{1}{2}
\begin{pmatrix*}[r]
1&-1\\
-1&1
\end{pmatrix*}
\times
\frac{1}{2}
\begin{pmatrix*}[r]
1&-1\\
-1&1
\end{pmatrix*}\\
&=
\frac{1}{4}
\begin{pmatrix*}[r]
2&-2\\
-2&2
\end{pmatrix*}\\
&=
\frac{1}{2}
\begin{pmatrix*}[r]
1&-1\\
-1&1
\end{pmatrix*}\\
&=
\hat{P}_-
\end{split}
\end{equation*}

\end{proof}

\[
\hat{P}_+\hat{P}_-=0
\]

\begin{proof}

\begin{equation*}
\begin{split}
\hat{P}_+\hat{P}_-&=
\frac{1}{2}
\begin{pmatrix*}[r]
1&1\\
1&1
\end{pmatrix*}
\times
\frac{1}{2}
\begin{pmatrix*}[r]
1&-1\\
-1&1
\end{pmatrix*}\\
&=
\frac{1}{4}
\begin{pmatrix*}
0&0\\
0&0
\end{pmatrix*}\\
& =
0
\end{split}
\end{equation*}
\end{proof}

\[
\hat{P}_-\hat{P}_+=0
\]

\begin{proof}

\begin{equation*}
\begin{split}
\hat{P}_-\hat{P}_+&=
\frac{1}{2}
\begin{pmatrix*}[r]
1&-1\\
-1&1
\end{pmatrix*}
\times
\frac{1}{2}
\begin{pmatrix*}[r]
1&1\\
1&1
\end{pmatrix*}\\
&=
\frac{1}{4}
\begin{pmatrix*}
0&0\\
0&0
\end{pmatrix*}\\
& =
0
\end{split}
\end{equation*}
\end{proof}

\clearpage

\begin{center}
\begin{tcolorbox}[title= 5.3.12 \hfill Isaac Baer , halign title=center]
\begin{center}
Let $A=
\begin{pmatrix*}[r]
4&2&2\\
2&4&2\\
2&2&4
\end{pmatrix*}$ with eigenvalues $\lambda=2,8$
\end{center}

\vspace{3mm}

Diagonalize A if possible. 
\end{tcolorbox}
\end{center}

\vspace{10mm}

In order to express A as the product of diagonal and invertible matrices, we must find bases for the eigenspaces of A.

For $\lambda=2$, the eigenspace is the null space of the matrix $(A-\lambda I)$

\[(A-\lambda I)=
\begin{pmatrix*}[r]
2&2&2\\
2&2&2\\
2&2&2
\end{pmatrix*}
\]

\vspace{1mm}

We can find this null space by row reducing $(A-\lambda I)$ augmented with the zero vector.

\[
\begin{pmatrix*}[r]
2&2&2&0\\
2&2&2&0\\
2&2&2&0
\end{pmatrix*}
\sim
\begin{pmatrix*}[r]
1&1&1&0\\
0&0&0&0\\
0&0&0&0
\end{pmatrix*}
\]

\vspace{3mm}

So elements of the eigenspace associated with 2 are of the form

\[X=a
\begin{pmatrix*}[r]
-1\\
1\\
0
\end{pmatrix*}
+b
\begin{pmatrix*}[r]
-1\\
0\\
1
\end{pmatrix*}
\]

And thus a basis for this eigenspace is 

\[
\Bigg\{
\begin{pmatrix*}[r]
-1\\
1\\
0
\end{pmatrix*}
,
\begin{pmatrix*}[r]
-1\\
0\\
1
\end{pmatrix*}
\Bigg\}
\]

For $\lambda=8$, 

\[
(A-\lambda I)=
\begin{pmatrix*}[r]
-4&2&2\\
2&-4&2\\
2&2&-4
\end{pmatrix*}
\]

Again we can row reduce the matrix augmented with the 0 vector

\[
\begin{pmatrix*}[r]
-4&2&2&0\\
2&-4&2&0\\
2&2&-4&0
\end{pmatrix*}
\sim
\begin{pmatrix*}[r]
-2&1&1&0\\
1&-2&1&0\\
1&1&-2&0
\end{pmatrix*}
\sim
\begin{pmatrix*}[r]
-2&1&1&0\\
0&-\frac{3}{2}&\frac{3}{2}&0\\
0&\frac{3}{2}&-\frac{3}{2}&0
\end{pmatrix*}
\sim
\begin{pmatrix*}[r]
-2&1&1&0\\
0&1&-1&0\\
0&0&0&0
\end{pmatrix*}
\sim
\begin{pmatrix*}[r]
1&0&0&0\\
0&1&-1&0\\
0&0&0&0
\end{pmatrix*}
\]

Now we see members of the eigenspace of 2 are of the form

\[
X=a
\begin{pmatrix*}
0\\
1\\
1
\end{pmatrix*}
\]

and therefore this vector can serve as a basis.

Now that we have all eigenvalues, and a basis large enough to span $\mathbb{R}^3$, we can express A as the product $PDP^{-1}$ where $D$ is the diagonal matrix with its diagonal entries equal the eigenvalues of A, and P is the matrix with columns which form the eigenbasis of $\mathbb{R}^3$. P and D are as follows.

\[
P=
\begin{pmatrix*}[r]
-1&-1&0\\
1&0&1\\
0&1&1
\end{pmatrix*}
\text{ and }
D=
\begin{pmatrix*}[r]
2&0&0\\
0&2&0\\
0&0&8
\end{pmatrix*}
\]

We can find $P^{-1}$ in the usual way, by row reducing $P$ augmented with $I$. We find 

\[
P^{-1}=
\begin{pmatrix*}[r]
-\frac{1}{2}&\frac{1}{2}&-\frac{1}{2}\\
-\frac{1}{2}&-\frac{1}{2}&\frac{1}{2}\\
\frac{1}{2}&\frac{1}{2}&\frac{1}{2}
\end{pmatrix*}
\]

Therefore

\[
A=
\begin{pmatrix*}[r]
-1&-1&0\\
1&0&1\\
0&1&1
\end{pmatrix*}
\begin{pmatrix*}[r]
2&0&0\\
0&2&0\\
0&0&8
\end{pmatrix*}
\begin{pmatrix*}[r]
-\frac{1}{2}&\frac{1}{2}&-\frac{1}{2}\\
-\frac{1}{2}&-\frac{1}{2}&\frac{1}{2}\\
\frac{1}{2}&\frac{1}{2}&\frac{1}{2}
\end{pmatrix*}
\]




\end{document}  